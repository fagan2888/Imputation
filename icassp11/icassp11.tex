% Template for ICASSP-2010 paper; to be used with:
%          spconf.sty  - ICASSP/ICIP LaTeX style file, and
%          IEEEbib.bst - IEEE bibliography style file.
% --------------------------------------------------------------------------
\documentclass{article}
\usepackage{spconf,amsmath,graphicx}

% Example definitions.
% --------------------
\def\x{{\mathbf x}}
\def\L{{\cal L}}

% Title.
% ------
\title{IMPUTATION OF BEAT-ALIGNED FEATURES FOR MUSIC PATTERN LEARNING}
%
% Single address.
% ---------------
%\name{Thierry Bertin-Mahieux\thanks{Thanks to NSERC and some other stuff.}}
%\address{EE dept., Columbia University}
%
% For example:
% ------------
%\address{School\\
%	Department\\
%	Address}
%
% Two addresses (uncomment and modify for two-address case).
% ----------------------------------------------------------
\twoauthors
  {Thierry Bertin-Mahieux\sthanks{NSERC}, Ron J. Weiss\sthanks{Thanks something}}
	{Columbia University / New York University\\
          LabROSA / MARL\\
	New York, USA}
  {Graham Grindlay, Daniel P.W. Ellis}
	{Columbia University\\
          LabROSA\\
	New York, USA}

\begin{document}
%\ninept
%
\maketitle
%
\begin{abstract}
Imputation is cool. It is well-defined problem, as opposed to segmentation.
Gives a reasonable benchmark to compare algorithms that claim learning meaningful
patterns on music. Hard to beat benchmarks comparison, for instance linear
prediction. We compare 6 methods on a large dataset.
\end{abstract}
%
\begin{keywords}
Missing data, chroma features, imputation, SIPLCA, codebook learning
\end{keywords}
%
\section{Introduction}
\label{sec:intro}

See abstract.

In this paper: definition of the task, presentation of the algorithms, experimental
results.

\section{RELATED WORK}
\label{sec:relatedwork}

Paris Smaragdis on speech data \cite{Smaragdis2009}.

Similarity with work on automatic music composition such as \cite{Mozer1994a} or music
expectation \cite{Hazan2010}, but we do not further consider these tasks in this work.

Learned patterns for music data recently includes: \cite{Bertin-Mahieux2010a,Casey2007,Weiss2010}.

\section{TASK DEFINITION}
\label{sec:task}
Beat-chroma features used in \cite{Ellis2007a}.


\section{METHODS}
\label{sec:methods}
In this section we present the different methods we use for imputing the missing data.
The ``trivial methods'' that do not require any learning, and simply use the available
data from the rest of the song to make an educated guess. We will later see that this
eudcated guess is remarkably good.

The learned models are of more interest for future research, even though they do not
necessarily perform better at the moment. Learning a model that predicts subsequent
notes or harmonic patterns can be applied to other tasks such as cover recognition,
segmentation, similarity and compression.

\subsection{TRIVIAL METHODS}
\label{ssec:trivialmethods}

Random, average, knn.

\subsection{LEARNED MODELS}
\label{ssec:learnedmodels}

Linear predictor, codebook, SIPLCA.

SIPLCA code available online \cite{Weiss2010}. Code to get Echo Nest features and perform
online clustering available online \cite{Bertin-Mahieux2010a}.

\section{EXPERIMENTS}
\label{sec:page}

\subsection{ERROR MEASURES}
\label{ssec:errmeasures}
We measure squared euclidean distance or symmetric Kullback-Leibler (KL) divergence,
both averaged over missing pixels. It is unclear which measure is the beat for features
as the ones we use. In \cite{Sajda2003}, author discuss the two for
a NMF method and show that they depend on the noise model that is assumed.
A poisson noise model (versus a gaussian one) seems to better fit audio perception, 
which result in the use of KL. See also \cite{Fevotte2009} for a comparison on a piano
excerpt.

\centering
\centerline{\includegraphics[width=\columnwidth]{imputation}}
\vspace{.1cm}
\centerline{Caption here, hein???}\medskip


\section{ILLUSTRATIONS, GRAPHS, AND PHOTOGRAPHS}
\label{sec:illust}

Illustrations must appear within the designated margins.  They may span the two
columns.  If possible, position illustrations at the top of columns, rather
than in the middle or at the bottom.  Caption and number every illustration.
All halftone illustrations must be clear black and white prints.  Colors may be
used, but they should be selected so as to be readable when printed on a
black-only printer.


\section{CONCLUSION AND FUTURE WORK}
\label{sec:conclusion}
Push the task as an evaluation for different learning method on such feature.
Way to unify different tasks.

Many algorithms remain to be tried, for instance large scale or deep recurrent
neural network, or using the nonlocal-means algorithm as has been done for
images \cite{Buades2005}.




% References should be produced using the bibtex program from suitable
% BiBTeX files (here: strings, refs, manuals). The IEEEbib.bst bibliography
% style file from IEEE produces unsorted bibliography list.
% -------------------------------------------------------------------------
\bibliographystyle{IEEEbib}
\bibliography{tbm_bib}

\end{document}
